\paragraph{Data Set and Descriptive Analysis}

The data set is made of 10 columns indicating the following variables: “Paz.id”, “log. lvmi”, “fuyrs”, “status”, “sex”, “age”, “con.cabg”, “creat”, “lv”, “sten.reg.mix”. The total of records is 255.
The first variable indicated as “Paz.id” is simply an identification number assigned to each patient in the study; the second variable “log.lvmi” stays for Natural Logarithm of Left Ventricular Mass Index, which is a measure of the mass of the heart’s left ventricle relative to the patient’s body size, this measure is an important indicator of cardiac health; the third column “fuyrs” is the follow-up in years of the patients from the surgery; the “status” indicates a binary variable whether an event has occurred or if they are consored, in our specific case dead = 1, censored = 0; “sex” of the patient is 0 if male and 1 if female; in additional we have the age of the patient at the time of the surgery; “con.cabg” indicates the presence of Coronary Bypass, this represents if the patient underwent coronary bypass surgery in addition to the heart valve implantation, 1 indicates presence and 0 the absence of coronary bypass; the variable “creat” is the serum creatinine level which indicates a measure for kidney function; “lv” stays for Left Ventricular Ejection Fraction which measures the blood the left ventricle pumps out with each contraction and the last column “sten.reg.mix” that describes the hemodynamic condition of the heart valve before surgery.
Here you can find a summary about the main information of the variables.
The range of the age goes from 23 to 89 years old.
The variables "lv" and "sten.reg.mix" each have three levels, representing different conditions related to heart function and valve hemodynamics.
For "lv", level 1 signifies a high ejection fraction, indicating good heart funtion; level 2 represents a moderate ejection fraction, possibly suggesting some impairment but still within normal range and level 3 indicates the worst scenario, so a low ejection fraction which can lead to symptoms of heart failure, for example. 
Regarding "sten.reg.mix", level 1 denotes stenosis, where the valve is narrowed, potentially reducing blood flow; level 2 indicates regurgitation, where the valve fails to close properly,  leading to blood leakage; and level 3 indicates a mixed condition, combining stenosis and regurgitation.
We have exploited the main information about each variables, such as the mean, standard deviation, the minimum and maximum values, also the interquartiles.
After having extracted the main information, the next important step was plotting a correlation matrix, in order to highlight already possible correlations between variables. We observed that there is correlation between the variables "status" and "age" (index of correlation equal to 0.33).