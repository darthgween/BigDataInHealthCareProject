\documentclass[12pt,a4paper]{report}

\usepackage[utf8]{inputenc}

\usepackage[titletoc,title,toc,page]{appendix}
\usepackage{verbatim}
\usepackage{placeins}
\usepackage{listings}
\usepackage[hidelinks]{hyperref}
\usepackage[english]{babel}
\usepackage{tikz}
\usepackage{parskip}
\usepackage{booktabs}
\usepackage{makeidx}


\usepackage{geometry}

\usepackage{graphicx}
\usepackage{blindtext}
\usepackage{chngcntr}
\counterwithin{table}{chapter}

\usepackage{newlfont}
\usepackage{fancyhdr}
\usepackage{indentfirst}
\usepackage{float}
\usepackage{hyperref}
\usepackage{soul}
\usepackage{titlesec}
\titlespacing{\chapter}{0pt}{-2cm}{0,5cm}

\usepackage[font=footnotesize,labelfont=bf]{caption}

\usepackage{multirow}
\usepackage[none]{hyphenat}

\usepackage{lscape} 

\usepackage{natbib}
\bibliographystyle{plainnat}
\setcitestyle{numbers,square,super}

\newcommand{\skipline}{\hfill \break \\}
\newcommand{\uscore}{\hspace{.1em}\rule[0em]{.0mm}{.3pt}}
\newcommand{\rom}[1]{\uppercase\expandafter{\romannumeral #1\relax}}

\titleformat{\chapter}[display]
  {\normalfont\bfseries}{}{0pt}{\Huge}

\usepackage{pdfpages}



\font\myfont=cmr12 at 25pt
\font\myfonti=cmr12 at 18pt
\documentclass{article}
\usepackage{graphicx} % Required for inserting images
\title{Big Data In Health Care: \\ Valve Heart}
\author{\\ \large Alessando Rota[798050] \vspace{0.5cm}\\  \large Yasmin Bouhdada [4354438] \vspace{0.5cm}\\ \large Matteo Mondini[902873] \vspace{0.5cm}\\ \large Alexandre Crivellari[902064]\vspace{0.5cm}\\}
\date{April 2024}

\begin{document}


\maketitle

\chapter{Introduction}
Heart Valve implantation is a medical treatment in patients to replace a damaged or diseased heart valve with an artificial or biological tissue valve. The valves play a crucial a role in regulating blood flow through heart chambers ensuring that blood is pumped in the right direction. Indeed, when the valve is damaged,  also the blood is less pumped in the right direction and the diseased valves need to be surgically repaired or replaced.  Now, according to medical articles and references, the surgery carries risks and several factors must be considered. Our work is to optimize and predict this risk percentage, highlighting which factors can have more impact and support thus the patient in making decision. The aim of this study is to compare the clinical outcomes and long-term survival in patients who underwent heart valve implantation, and to investigate the factors that can have an effect on the outcome of the patient. In particular, our study analyzes how the variables that compose the data set may impact the patient survival, based on both demographic information, such as age and sex, and pre-operative health status, which can intuitively be relevant in terms of patient at risk. So, we focused on a preliminary explanatory analysis in order to highlight correlations between variables, extract the main information and have an overview of the clinical course, object of our work.
For this purpose, it was necessary to discuss about the context and get familiar to the clinical concepts expressed from data set, so we could be more conscious and aware of the importance of specific numbers and levels; especially to interpret the results and identify interesting insights.



\newpage 

\end{document}